\clearpage{\pagestyle{empty}\cleardoublepage}

\chapter*{Introduction}\label{chap:intro}

\vskip-2.5cm

%\subsection*{Part I}
July 4th, 2012, is a historical date for high energy physics,
being the day when the ATLAS and CMS experiments at CERN announced
the discovery of a new boson consistent with a Standard Model Higgs
boson with mass $m_H\sim 125\gev$. Is it going to be celebrated, 20 years
from now, as the beginning of a new era of discoveries or as the 
end of the adventure? Is there something {\it more}, out there, in 
the outer space or 100~m underground, awaiting for being observed?
As outlined in Chapter~\ref{chap:TH} there is quite evidence
something must be there lying ``beyond the Standard Model''. A
successful theory, the Standard Model as it is, finally completed
by the identification of the Higgs boson, still leaves too many questions
unanswered. What is ``Dark Matter'', this kind of matter that does not
behave like atoms, being immune to electromagnetic interactions, but which
is known to account for $\sim$27\% of the total matter in the Universe?
After the Big Bang, what caused the asymmetry in particles and
antiparticles production that made matter prevail on antimatter?
Why is the top quark so much heavier than the other quarks? Why is the
Higgs boson so much lighter than the Planck mass?

It was to find an explanation to this puzzle that the 
Large Hadron Collider project initiated 20 years ago. The ATLAS
collaboration then started the design of the detector described in
Chapter~\ref{chap:atlas}, outlining a challenging physics program
in which the search for the Higgs boson was ``just'' the first bullet
of the list.
The LHC first run on the 20th of November 2009, with the ATLAS
experiment beginning to record data from these early proton-proton
collisions at a \cme\ of $900$~GeV just three days later.
Since then outstanding performances of both the hadron collider
and the detector allowed to collect a huge amount of data
from proton-proton collisions at increasing \cme s, reaching in
2012 a total of $\sim$20~\ifb\ at a \cme\ of 8~TeV.

%During the long time that passed between the detector construction
%and when the real data became available in 2009, analyses relied on
%data collected during test-beam runs and on Monte Carlo simulation.
Data alone would not tell much if it could not be possible to
compare them to the Standard Model prediction.
Chapter~\ref{chap:mc} illustrates the Monte Carlo techniques used
to obtain simulated samples of either Standard Model processes or
new physics events. 
%Monte Carlos are powerful tools that used with the purpose of 
%calibrating the detector, 
%modeling the contributions to some particular channel
%allow 
Starting from the computation of the matrix element of a 
particular process cross section, Monte Carlo tools
are combined to obtain the complete picture of how the
event of interest develops, including as a last step
the simulation of the particles interactions with the
detector material.

Being real data or Monte Carlo simulated samples, at
the ``raw'' level events are simple digital outputs coming
from the real or simulated response of the read-out electronics
of the ATLAS detector various subsystems. How these outputs
are assembled into physical objects is described in Chapter~\ref{chap:objects},
where the reconstruction process is explained.
The outcome is a dataset containing all the information needed
about leptons, jets and energy imbalance of the event,
ready to be processed by analyses.


%\subsection*{Part II}
%The object of this dissertation is then introduced in Chapter~\ref{chap:vlq}.
Motivated by more than one theoretical proposals for beyond Standard Model
physics, the Exotics group of the ATLAS collaboration set a search strategy 
for heavy quarks different from the first three generations in that they
do not show a chiral behavior under the electroweak group transformations.
These quarks are called ``vector-like'' and even though they are
predicted in various context, like extra-dimensions or composite Higgs
models, no details on their masses, nor on their decay branching fractions
are given. Searches must therefore rely the minimum as possible on 
assumptions from the model, and this is approach chosen for the
two searches in the single lepton channel
for pair-produced heavy vector-like top partners 
presented in this dissertation. The general quasi-model independent
strategy common to the two analyses, performed analyzing 
$\sim$14~\ifb\ of data from proton-proton collisions at the
 \cme $\rts=8\tev$ recorded during the year 2012 
at the ATLAS experiment, is presented in
Chapter~\ref{chap:vlq}.

The search for pair-produced heavy vector-like top partners
where at least one of them decays into a $W$ boson and a bottom
quark is detailed in Chapter~\ref{chap:wbx}. The key point in
this analysis is the reconstruction of the $W$ boson from its
hadronic decay products which allows for the reconstruction
of the heavy quark mass, a very good discriminant between
signal and background Standard Model contributions to the
search channel.

Chapter~\ref{chap:htx} then overviews the search for 
pair-produced heavy vector-like top partners
where at least one of them decays into a Standard Model Higgs
boson and a top quark. In this case the main decay of the Higgs
boson into two bottom quarks is exploited to define 
final channels characterized by a high number of recontructed
jets with also a high number of them identified as jets coming
from the hadronization of a bottom quark by a so-called
``\btag ging algorithm''.

While the individual results from the searches
are presented in the respective chapters, higher
sensitivity is achieved combining the two analyses.
This is described in Chapter~\ref{chap:results},
where also a prospect is given of a potential combination
of these searches with the other analyses performed
by ATLAS searching for heavy vector-like quarks
in final states with two leptons.






%\newpage
%\phantomsection
%\addcontentsline{toc}{section}{Acknowledgement}
\subsubsection*{Personal contributions and acknowledgement}

The results presented in this dissertation represent
a small sample of the achievements made possible by
the combined effort of many, many people. The ATLAS
collaboration itself consists of $\sim$3000 scientists,
working in different subgroups. 
Regarding the commitments for the ATLAS community, the 
author of this dissertation participated to calibration
and performance studies of the hadronic calorimeter
and to the improvement of data-driven estimation of
multi-jet backgrounds for analyses with top quarks 
decaying in the ``muon and jets'' channel. Regarding
the two analyses object of this dissertation, the author
has been amongst the main analyzers, implementing and
running the signal selection and statistical analysis and
performing the needed cross checks for data to background
modeling.% and for Monte Carlo signal modeling.

Finally, a particular acknowledgement goes to the ``IFAE-top''
group, present and former members, for their fundamental
contributions to the common analysis frameworks used
for these searches.
