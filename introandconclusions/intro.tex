\clearpage{\pagestyle{empty}\cleardoublepage}

\chapter*{Introduction}\label{chap:intro}

%\vskip-2.5cm

%\subsection*{Part I}
July 4th, 2012, represents a milestone for high-energy physics,
being the date when the ATLAS and CMS experiments at CERN announced
the discovery of a new particle consistent with a Standard Model Higgs
boson with mass $m_H\sim 125\gev$. Is it going to be celebrated, 20 years
from now, as the beginning of a new era of discoveries or as the 
end of the adventure? Is there something {\it more}, out there, in 
the outer space or 100~m underground, awaiting for being discovered?
As outlined in Chapter~\ref{chap:TH} there is quite some evidence
something must be there lying ``beyond the Standard Model''. 
A successful theory finally completed
by the identification of the Higgs boson, 
the Standard Model as it is still leaves too many questions
unanswered. What is ``Dark Matter'', this exotic form of energy density different
from atoms, being immune to electromagnetic interactions, but which
is known to account for $\sim$27\% of the total matter in the Universe?
After the Big Bang, what caused the asymmetry in the production of particles vs
antiparticles that made matter prevail over antimatter?
Why is the top quark so much heavier than the other quarks? Why is the
Higgs boson so much lighter than the Planck mass?

It was to find an explanation to this puzzle that the 
Large Hadron Collider project was initiated 20 years ago. The ATLAS
collaboration then started the design of the detector described in
Chapter~\ref{chap:atlas}, outlining an ambitious physics program
in which the search for the Higgs boson was ``just'' the first bullet
of the list.
The LHC first run was on the 20th of November 2009, with the ATLAS
experiment beginning to record data from these early proton-proton
collisions at a \cme\ of $900$~GeV just three days later.
Since then, outstanding performances of both the accelerator
and the detector allowed to collect a huge amount of data
from proton-proton collisions at increasing center of mass energies, reaching in
2012 a total of $\sim$20~\ifb\ at a \cme\ of 8~TeV.

%During the long time that passed between the detector construction
%and when the real data became available in 2009, analyses relied on
%data collected during test-beam runs and on Monte Carlo simulation.
However, this large amount of data alone would 
not tell much if it were not possible to
compare them to precise theoretical predictions.
Chapter~\ref{chap:mc} describes the Monte Carlo techniques used
to obtain simulated samples of either ``Standard Model'' %processes 
or ``new physics'' events. 
%Monte Carlos are powerful tools that used with the purpose of 
%calibrating the detector, 
%modeling the contributions to some particular channel
%allow 
Starting from the computation of the matrix element of a 
particular process, Monte Carlo tools
are combined to obtain the complete picture of how the
event of interest develops, including as a last step
the simulation of the particles interactions with the
detector material.

Whether real data or Monte Carlo simulated samples, at
the ``raw'' level events are simple digital outputs, coming respectively
from the real or simulated response of the read-out electronics
of the different ATLAS detector subsystems. How these outputs
are assembled into physical objects is described in Chapter~\ref{chap:objects},
where the reconstruction process is explained.
The outcome is a dataset containing all the information needed
about physics objects such as leptons, jets and energy 
imbalance of the event,
ready to be processed by analyses.


%\subsection*{Part II}
%The object of this dissertation is then introduced in Chapter~\ref{chap:vlq}.
Using these kind of datasets, 
%motivated by more than one theoretical proposals for beyond Standard Model physics, 
the Exotics group of the ATLAS collaboration defined a search strategy 
for exotic heavy quarks different from the first three generations 
%in that they do not show a chiral behavior under the electroweak group transformations.
%These quarks are 
and called ``vector-like''. Even though these quarks are
predicted in various proposed extentions of the Standard Model, 
like extra-dimensions or composite Higgs
models, no details on their masses are given and their decay branching fractions
are very model dependent. Searches aiming at inclusivity 
must therefore rely as little as possible on 
assumptions from the model, and this is the approach chosen for the
two searches in the single lepton channel
for pair-produced heavy vector-like top partners 
presented in this dissertation. The general quasi-model independent
strategy common to the two analyses, performed analyzing 
$\sim$14~\ifb\ of data from proton-proton collisions at the
 \cme\ $\rts=8\tev$ recorded during the year 2012 
at the ATLAS experiment, is presented in
Chapter~\ref{chap:vlq}.

The search for pair-produced heavy vector-like top partners
where at least one of them decays into a $W$ boson and a bottom
quark is detailed in Chapter~\ref{chap:wbx}. The key point in
this analysis is the reconstruction of the $W$ boson from its
hadronic decay products which allows for the reconstruction
of the heavy quark mass, a very good discriminating variable between
signal and Standard Model background processes.

Chapter~\ref{chap:htx} presents the search for 
pair-produced heavy vector-like top partners
where at least one of them decays into a Standard Model Higgs
boson and a top quark. In this case the main decay of the Higgs
boson into two bottom quarks is exploited resulting in a
final state signature characterized by a high number of recontructed
jets, where a large fraction of them is identified as originating
from the hadronization of bottom quarks.

While the individual results from the searches
are presented in the respective chapters, higher
sensitivity is achieved combining the two analyses.
This is described in Chapter~\ref{chap:results},
and the result of these searches is compared with
other similar searches exploiting multi-lepton signatures.
%where also a prospect is given of a potential combination of these searches with the other analyses performed by ATLAS searching for heavy vector-like quarks in final states with two leptons.






%\newpage
%\phantomsection
%\addcontentsline{toc}{section}{Acknowledgement}
\subsubsection*{Personal contributions and acknowledgement}

The results presented in this dissertation represent
a small sample of the achievements made possible by
the combined effort of many, many people. The ATLAS
collaboration itself consists of $\sim$3000 scientists,
working in different subgroups. In particular,
the author of this dissertation participated to calibration
and performance studies of the hadronic calorimeter
and to the improvement of data-driven estimation of
multi-jet backgrounds for analyses with top quarks 
decaying in the single-muon channel. Regarding
the two analyses object of this dissertation, the author
has been amongst the main analyzers, implementing and
running the signal selection and statistical analysis and
performing the needed cross checks for the good modeling
of background predictions. % and for Monte Carlo signal modeling.
The results are documented in two preliminary 
notes~\cite{ATLAS-CONF-2013-060,ATLAS-CONF-2013-018}.
The work for the final analyses to be published is still on-going at
the time of the writing of this dissertation.
The author also significantly participated to a previous,
published analysis~\cite{ATLAS:2012qe}
performed on the data from lower \cme\ 
proton-proton collisions which pioneered searches for
heavy vector-like quarks in ATLAS.

A particular acknowledgement goes to the ``IFAE-top''
group, present and former members, for their fundamental
contributions to the common analysis framework used
for these searches.
