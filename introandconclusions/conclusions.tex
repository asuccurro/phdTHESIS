\clearpage{\pagestyle{empty}\cleardoublepage}

\chapter*{Conclusions and outlook}\label{chap:conclusions}

Two quasi-model independent searches for vector-like top partners 
pair-produced in proton-proton collisions at a \cme\ of 8~\tev\ 
have been presented in this dissertation. The final states considered
for both analyses involve one lepton and many jets but different
strategies are adopted in order to achieve sensitivities in different
corners of the decay phase space. Indeed, a peculiar fact for these
searches that has been stressed many times over these pages is the 
unpredicted nature of the heavy vector-like top partners model. 
As a direct consequence, the two analyses have been designed
and developed to be optimized for a particular decay mode and
to have orthogonal final signal rich search channels, to allow
for a combined search that could exploit the specific sensitivities.

Three particular models, interesting from a theoretical point
of view (but not for this more favoured than others), are considered
over the two analyses: the chiral fourth-generation, with 
BR$(T\to Wb)=1$ for any value of the heavy quark mass; 
the singlet vector-like, with BR$(T\to Wb)\sim 0.5$ and 
BR$(T\to Ht)\sim 0.3$ for almost all
values of the heavy quark mass considered in the searches;
the doublet vector-like, with BR$(T\to Wb)= 0$ and 
BR$(T\to Ht)\in$ [0.50, 0.75] for all the values of the heavy quark mass.
In the \wbx\ analysis, it was possible to exclude at a 95\% CL
pair-produced chiral fourth-generation top partners and vector-like
$Y$ quarks with masses up to 740~\gev, and pair-produced vector-like 
singlet top partners with  masses up to 505~\gev.
In the \htx\ analysis, it was possible to exclude at a 95\% CL
pair-produced vector-like singlet and doublet top partners with 
masses up to 640~\gev\ and 790~\gev\ respectively.
When the two analyses are combined, the observed exclusion limit
for the only model where both analyses are sensitive, the
vector-like singlet $T$, is pushed $\sim$30~\gev\ further the
best result of the two, obtained by the \htx\ analysis,
achieving a 95\% CL exclusion of pair-produced vector-like singlet 
top partners with masses up to 670~\gev. While this might not
look like an incredible improvement, the power of the combination
of the two searches is evident looking at the coverage of the
two-dimensional branching ratio plane, where  95\% CL exclusion is set for
pair-produced vector-like top partners with masses up to 550~\gev\ 
independently from the model, and also the plane for the 600~\gev\ mass
point is almost fully excluded. This strongly encourages to perform,
in the future, full combination of searches for vector-like quarks.

The mass range excluded at 95\% CL up to now
is getting closer and closer to the point where pair-production
of vector-like quarks will start to be disfavoured with respect
to single production. In this sense, while it is desirable to
exploit the know-how achieved up to now with the searches for
pair-produced vector-like quarks in the single lepton and dilepton
channels, it is a good idea to start designing searches for
single-produced vector-like quarks for LHC Phase-II.
Further improvements are possible for the searches
presented in this dissertation without changing the core of
the analysis strategies. During Phase-II they will benefit
of the increased \cme\ available for heavy quark production
in pp collisions with \rts=14~\tev\ and of the high integrated
luminosity (100~\ifb\ of data are expected over three years
of operation). With high luminosity comes the challenge of
dealing with higher pile-up, but considering that vector-like
quark searches involve high-\pt\ objects this should not
represent a major issue.

Besides the great discovery potential, the combination
of multiple searches will provide useful insights on the
exotic quark properties, like their quantum numbers or
the measurement of their branching ratios. 
Adding searches for single production of vector-like quarks
would also allow to measure the electroweak couplings of
these particles with the Standard Model quarks from the third generation.
Finally, these searches are even more interesting since
they will probe a wide range of
signatures that are often shared with other new physics
scenario. Given the fact that during these last successful years
of LHC operation no hints on what lies ``beyond the Standard Model''
have emerged, the winning strategy is for sure not to confine ourselves 
to exclusive models.
