\clearpage{\pagestyle{empty}\cleardoublepage}

\chapter{Going beyond the Standard Model}\label{chap:TH}

The Standard Model (SM) of particle physics is the most successful, beautyful and precise theory describing the interactions
between fundamental particles. Its validity has been tested by precision measurements at the Large Electron-Positron Collider (LEP)
at CERN and confirmed by the observation of all the particles it predicts, including the Higgs-like boson discovered at the
Large Hadron Collider (LHC) in July of 2012 which up to now behaves as expected from the SM.

What makes the SM ``only'' and effective theory is the fact that unstabilities appear at high energy scales of the order of the
Planck mass. In this Chapter we will show 

\section{Building the Standard Model}\label{sec:THsm}


\section{New Physics Models predicting vector-like quarks}\label{sec:THvlq}

\cite{AguilarSaavedra:2009es,Martin:2009bg}
