\clearpage{\pagestyle{empty}\cleardoublepage}

\chapter{Preliminary search for \TTbar\ pairs decaying to $Wb+X$}\label{chap:wbx}

\section{Boosted $W$ reconstruction}\label{sec:boostedW}

\section{Control regions}\label{sec:wbxCR}

\section{Event selection}\label{sec:wbxEVT}

%\section{}\label{sec:}

%\section{}\label{sec:}

\section{Systematics}\label{sec:wbxSYS}

\subsection{Technical Details}

\subsubsection{Merging of non-$t\bar{t}$ Backgrounds}

Due to the very low statistics of the remaining backgrounds ($W+$jets, $Z+$jets, dibosons, single top, $t\bar{t}V$), we are merging all of them into a single component as it will prevent us from pathological behavior in the treatment of some systematic uncertainties, $e.g.$ when a bin is empty in the nominal case, while an entry appears in a variation. The QCD background prediction from the Matrix Method is negative (but consistent with zero) and is currently being set to zero. The total of all these backgrounds is referred as ``non-$t\bar{t}$" in the following. 

This treatment is believed to be reasonable as we do not expect or see any resonance for these backgrounds. Specific systematic variations of each background, $e.g.$ uncertainties on cross-section normalization, are applied separately to each corresponding component, and then added to the others in the nominal case to form a variational histogram of the sum. This can be seen on Appendix~\ref{app:ShapeSystematics}.

%Contrary to what was described in Table~\ref{tab:sys}, all systematics are considered as normalization only for this small-background component. For the {\sc Tight} selection, we take the values derived from the {\sc Loose} selection, as we do not expect significant difference between the two cases.

\subsubsection{Interpolation Method for Systematic Uncertainties}
The standard method implemented in {\sc mclimit} is to interpolate the histograms between the nominal and the shifted templates. 
Thus a shift of $+0.5\sigma$ will correspond to half way between the nominal and $+1\sigma$ shifted template. This is carried out bin-by-bin.
Such an interpolation method is referred to as vertical morphing using a linear interpolation. Some of the uncertainties produce very 
asymmetric effects. We use quadratic interpolation to ensure a continuous derivative at zero shift for variations below $1\sigma$ and 
linear interpolation above. We apply interpolation of all systematic uncertainties and generate pseudo-experiments using these 
interpolated numbers.

\subsubsection{Statistical uncertainty of the MC samples}
The statistical uncertainty of the MC samples is taken into account when computing our likelihoods.
The {\sc mclimit} flag used is 2, meaning we consider the uncertainties as given by the uncertainties of the templates. 
This allows us to correctly estimate the statistical uncertainty when many templates are merged together with different weights, such as in
the case of the merged non-$t\bar{t}$ background.

\subsubsection{Profiling of Systematic Uncertainties}
This analysis does not profile any of the fundamental sources of systematic uncertainty. The only profiled parameter
in the {\sl loose} analysis is the overall $t\bar{t}$ yield, taking advantage of the background-dominated
low $M_{reco}$ sideband region, which absorbs the cumulative effect of different possible sources of uncertainty
affecting the $t\bar{t}$ background normalization. Currently this parameter is free floating (since we still need to fix the
JES systematic for signal) but eventually it will be assigned as prior uncertainty to the sum in quadrature of all normalization 
uncertainties (both flat and shaped) affecting the $t\bar{t}$ background (was 24\% in the previous analysis). 
No profiled parameters are used in the {\sl tight} analysis.







\begin{table}[htb]
\centering
\caption{List of all systematic uncertainties (in \%) considered in the analysis, indicating which ones are treated
as normalisation and/or shape uncertainties, with their impact on normalisation in the case of the 
{\sl tight} selection, for signal and backgrounds. The signal given here is a chiral fourth-generation $T$ quark with mass $m_{T}=600\gev$.}
\begin{tabular}{l*{3}{c}}
\hline\hline
 & $\T\bar{\T}$ ($600\gev$) \rule{0pt}{2.6ex} \rule[-1.2ex]{0pt}{0pt} & $t\bar{t}$ & Non-$t\bar{t}$\\
\hline
\multicolumn{4}{c}{Uncertainties [\%] affecting only the normalisation of the $m_{\rm reco}$ distribution:}\\
Luminosity & +3.6/-3.6 & +3.6/-3.6 & +3.6/-3.6\\
Lepton trigger, reconstruction and ID efficiency & +2.0/-2.0 & +2.0/-2.0 & +2.0/-2.0\\
$t\bar{t}$ cross section & -- & +10/-11 & --\\
\hline
\multicolumn{4}{c}{Uncertainties [\%] affecting both normalisation and shape of the $m_{\rm reco}$ distribution:}\\
Jet energy scale & +6.6/-8.4 & +15/-15 & +33/-22\\  
Jet energy resolution & +8.4/-8.4 & +3.6/-3.6 & +9.3/-9.3\\ 
Jet identification efficiency & +2.3/-2.7 & +2.3/-2.5 & +1.9/-2.6\\  
$b$-quark tagging efficiency & +6.7/-7.3 & +6.7/-8.9 & +1.8/-2.2\\  
$c$-quark tagging efficiency & +1.6/-1.6 & +4.1/-4.1 & +5.6/-5.6\\  
Light-jet tagging efficiency & +0.3/-0.3 & +0.7/-0.7 & +2.7/-2.7\\  
$t\bar{t}$ modelling: NLO MC generator & -- & +48/-48 & --\\  
$t\bar{t}$ modelling: parton shower and fragmentation & -- & +25/-25 & --\\  
$t\bar{t}$ modelling: initial and final state QCD radiation & -- & +8.8/-8.8 & --\\   
$W$+jets normalisation & -- & -- & +8.9/-7.8\\  
$W$+heavy-flavor fractions & -- & -- & +18/-19\\  
$W$+jets modelling: scale variation & -- & -- & +11/-11\\  
$Z$+jets cross section & -- & -- & +1.1/-1.1\\ 
Single top cross section & -- & -- & +1.9/-1.5\\  
Diboson cross section & -- & -- & $<0.1\%$\\  
$t\bar{t}V$ cross section & -- & -- & +1.5/-1.5\\  
\hline
Total & +14/-15 & +59/-59 & +42/-35\\
\hline\hline
\end{tabular}
%\caption{List of all systematic uncertainties (in \%) considered in the analysis, indicating which ones are treated
%as normalisation and/or shape uncertainties, with their impact on normalisation in the case of the 
%{\sl tight} selection, for signal and backgrounds. The signal corresponds to a chiral fourth-generation $\T$ quark with mass $m_{\T}=600\gev$.}
\label{tab:SystSummary}
\end{table}
