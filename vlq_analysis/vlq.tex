\clearpage{\pagestyle{empty}\cleardoublepage}

\chapter{Searches for vector-like top partner pairs in the single lepton channel}~\label{chap:vlq}

In the following Chapter we will describe two searches for vector-like 
top partners \TTbar\ pairs performed in the single 
lepton\footnote{From now on, with the word ``lepton'' we will 
mean only either electron or muon, assumed to come from the leptonic
decay of a $W$ boson with its associated neutrino, which is considered
to be the only particle contributing to the transverse missing energy \met.} channel. 
These analyses
are optimized for different final states and are thus complementary.
The first search focuses on decay channels with high BR to $Wb$ and is 
performed using the full dataset of p-p collisions at the \cme of \rts=8~\tev\
collected during 2012 at the ATLAS detector, consinsting in 20.34~\ifb, while
the preliminary search for vector-like top partners with high BR to $Ht$
uses a partial dataset of the same data, amounting to 14.3~\ifb.

The Chapter is organized as follows: first Section~\ref{sec:presel}
introduces the common event preselection for data and few general concepts in the
analyses design; Section~\ref{sec:MCbkg}
presents the Monte Carlo samples used in the searches, which
are in general common to both analyses with only few exceptions that are reported;
Section~\ref{sec:qcdbkg} describes how the multi-jet background from QCD events is
obtained.
Finally, the two analyses are detailed in Section~\ref{sec:wbx} and Section~\ref{sec:htx}, 
which illustrate the event selection criteria, the background modeling estimation, 
the systematics affecting the analysis, the statistical treatment and the
results.

\section{Data sample and common event preselection}\label{sec:presel}

The data from p-p collision events recorded at the ATLAS experiment during
2012 at a \cme\ of $\rts=8\tev$ are considered. Physics object definitions 
were previously discussed in Section~\ref{sec:objects}.
Events collected during
stable beam periods are required to pass data quality requirements and
single lepton trigger selection. In order to maximize trigger
efficiency, different transverse momentum threshold triggers are combined
through a logical \OR, with the lower \pt\ ones including isolation requirements
that result in inefficiencies for high \pt\ lepton candidates, recovered with
the use of the higher threshold triggers. The electron triggers have
\pt\ thresholds of 24 and 60\gev, the muon ones of 24 and 36\gev.

After passing trigger requirements, events with more than one lepton are
discarded. In addition, the only lepton of the event has to match within $\dr<0.15$ the
triggered one. As basic preselectio, four jets satisfying the conditions
described in Section~\ref{sec:jets} are required, at least one of them
being tagged as a \bjet.

In order to suppress the multijet background from QCD processes,
combined cuts on the \met\ and on the tranverse mass of the 
leptonically decaying $W$ boson \mt\footnote{$\mt = \sqrt{2 p^\ell_{\rm T} \met (1-\cos\Delta\phi)}$, with
$p^\ell_{\rm T}$  being the transverse momentum (energy) of the muon (electron) and $\Delta\phi$ the
azimuthal angle separation between the lepton and the direction of
the missing transverse momentum.}\ 
are defined: $\met>20~\gev$ and $\met+\mt>60~\gev$.

At this point, a simple consideration about the typical expected jet
(and \bjet) multiplicity is made so as to define an orthogonality
cut between the two analyses. Table~\ref{tab:jetmult} shows the 
number of jets (\bjet s) per decay channel combinations of \TTbar\ pairs, 
in the case of single lepton selection with at least four jets
(i.e. one $W$ boson will always decay into lepton and neutrino,
and $Z$ boson decay to neutrinos is excluded in the $WbZt$ channel) and assuming that
the Higgs boson decays to a bottom quark-antiquark pair.
To avoid overlap between selected events from the two analyses, in the
\wbx\ analysis events with $\geq$6 jets and $\geq$3 \bjet s are 
rejected\footnote{As will be explained later in Section~\ref{sec:htxEVT}, another orthogonality
cut will be applied in the low \bjet\ multiplicity channel of the \htx\ analysis.}.

\begin{table}\centering
	\begin{tabular}{lccc}\toprule
	 & $Wb$ & $Ht$ & $Zt$ \\\midrule
             &\cellcolor{lightgray} & & \cellcolor{lightgray}\\
	\multirow{-2}{*}{$Wb$} & \cellcolor{lightgray}\multirow{-2}{*}{\bf 4 (2)} & \multirow{-2}{*}{6 (4)} & \cellcolor{lightgray}\multirow{-2}{*}{{\bf 6} ({\bf2}/4)} \\
        \multirow{2}{*}{$Ht$} & \multirow{2}{*}{6 (4)} & \multirow{2}{*}{8 (6)} & max: 8 (4/6)\\
             & & & \cellcolor{lightgray}min: {\bf6 (2)}\\
        \multirow{2}{*}{$Zt$} & \cellcolor{lightgray}& max: 8 (4/6) & \cellcolor{lightgray}max: {\bf8} ({\bf2}/6) \\
             & \cellcolor{lightgray}\multirow{-2}{*}{\bf6 (2/4)} & \cellcolor{lightgray}min: {\bf6 (2)} & \cellcolor{lightgray}min: {\bf6} ({\bf2}/4)\\
	\bottomrule\end{tabular}\caption{Jets (\bjet s) multiplicities in the various possible final states. $Z$ boson decays 55\% hadronically, 15\% of the 
        times into \bbbar, therefore the min/max number of \bjet s is reported. Highlighted are the channels that after the orthogonality cut
        will contribute to the \wbx\ analysis.}\label{tab:jetmult}
\end{table}




\section{Background and signal modeling}\label{sec:datasets}

The main background for both analyses is $t\bar{t}$ 
production with jets ($t\bar{t}$+jets in the following) 
and different choices for the generator are made
in the analyses because of the specific needs of having well
modeled regions.
In the case of the $t\bar{t}$+jets background prediction for the \htx\ analysis 
further corrections to match the data are applied, due to a mismodeling in the
heavy- and light-flavour content of the simulated sample (see Section~\ref{sec:htxEVT}).

$W$ boson production  in association with jets ($W$+jets in the following) 
and multijet events from QCD processed also contributes, the latter
sneaking into the event selection via the misidentification of a jet or a photon as an
electron or the presence of a non-prompt lepton from, e.g., semileptonic $b$- or $c$-hadron decay.
Other background smaller components are single top quark, $Z$+jets, diboson
($WW,WZ,ZZ$), and $t\bar{t}$ production associated with a vector or Higgs boson.

All event generators using {\sc Herwig}~\cite{herwig} are also interfaced to {\sc
Jimmy} v4.31~\cite{jimmy} to simulate the underlying event.  
With the exception of the 
signal samples, all simulated 
samples utilise {\sc Photos 2.15}~\cite{PhotosPaper} to model
photon radiation and {\sc Tauola 1.20}~\cite{TauolaPaper} to model
$\tau$ decays.  

All simulated samples include multiple p-p
interactions and go through the  {\sc Geant4}~\cite{geant}
detector geometry and response simulation~\cite{atlas_sim}
with the exception of the signal samples, for which a fast simulation of
the calorimeter response is used.

All simulated samples are then processed through the same reconstruction 
software as the data and are reweighted to match 
the instantaneous luminosity profile in data. 

%Additional corrections are applied so that the 
%object identification efficiencies, energy
%scales and energy resolutions match those determined in data control
%samples.


\subsection{Monte Carlo simulated samples}\label{sec:MCbkg}

\subsubsection{$t\bar{t}$ MC@NLO}\label{subsec:MC@NLO}
Simulated samples of $t\bar{t}$ pair production  in association with jets 
($t\bar{t}$+jets or simply $t\bar{t}$ in the following)
are generated with {\sc MC@NLO} v4.01~\cite{mcatnlo_1,mcatnlo_2,mcatnlo_3} using the {\sc CT10} set of parton distribution functions (PDFs)~\cite{ct10},
with the parton-shower and fragmentation steps being performed by 
{\sc Herwig} v6.520~\cite{herwig}.
The top quark mass is assumed to be equal to $172.5\gev$ and 
the samples are normalized to approximate next-to-next-to-leading-order 
(NNLO) theoretical cross section~\cite{ttbarxs}; the cross section used 
has been computed with {\sc Hathor} 1.2~\cite{ttbarxs} using the {\sc MSTW2008}
NNLO PDF set~\cite{mstw} and is $\sigma_{t\bar{t}}= 238^{+22}_{-24}$~pb, 
where the total uncertainty results from the sum in quadrature of the 
scale and PDF+$\alpha_S$ uncertainties according to 
the {\sc MSTW} prescription~\cite{mstw2}. 
This is the $t\bar{t}$ used in the \wbx\ analysis.

\subsubsection{$t\bar{t}$ Alpgen}\label{subsec:alpgen}
Simulated samples of $t\bar{t}$+jets are generated using
%and $W/Z$+jets events are generated using
 the {\sc Alpgen v2.13}~\cite{alpgen} leading-order (LO) generator and the 
{\sc CTEQ6L1} PDF set~\cite{cteq6}, with parton shower and fragmentation  
modelled through {\sc Herwig} v6.520~\cite{herwig}.

A parton-jet matching scheme called ``MLM matching''~\cite{mlm} is used
in orderd to avoid double-counting  of partonic configurations
eventually generated both at the matrix-element calculation level
and at the parton-shower evolution step.

Separate samples are generated for $t\bar{t}$+light jets ($t\bar{t}$+light 
or $t\bar{t}$+LF in the following, from ``light flavour'') 
with up to three additional light partons ($u$, $d$, $s$ quarks or gluons),
and for $t\bar{t}$+heavy-flavour jets ($t\bar{t}$+HF in the following), 
including $t\bar{t}b\bar{b}$ and
$t\bar{t}c\bar{c}$.  
An algorithm based on the angular separation
between the extra heavy quarks is used to remove 
the overlap between $t\bar{t}q\bar{q}$ ($q=b,c$) 
generated from the matrix element calculation and 
from parton-shower evolution in the  $t\bar{t}$+light samples
is employed: matrix-element prediction is chosen over the parton-shower one
when $\Delta R(q,\bar{q})>0.4$, else vice-versa.

%The algorithm used is implemented in the HFOR tool~\cite{hfor}.

Again a top quark mass of $172.5\gev$ is assumed, and normalisation to the
NNLO theoretical cross section is used (see~\ref{subsec:MC@NLO})

\subsubsection{$W/Z$+jets}

Simulated samples of $W/Z$ boson production in association with jets
($W/Z$+jets in the following) are generated with up to five additional 
partons using the {\sc Alpgen v2.13}~\cite{alpgen} LO generator and the 
{\sc CTEQ6L1} PDF set~\cite{cteq6}, interfaced to {\sc Herwig} v6.520 
for parton showering and fragmentation.

``MLM matching'' is used also here to avoid double-counting of partonic configurations 
between  matrix-element  calculation and parton showering.

The $W$+jets samples are generated separately for $W$+light jets, 
$Wb\bar{b}$+jets, $Wc\bar{c}$+jets, and $Wc$+jets, 
with the relative contributions normalized using the fraction 
of $b$-tagged jets in $W$+1-jet and $W$+2-jets data 
control samples~\cite{whf}, while
the $Z$+jets samples are generated separately 
for $Z$+light jets, $Zb\bar{b}$+jets, and $Zc\bar{c}$+jets and
normalized to the inclusive NNLO theoretical cross section~\cite{vjetsxs}.

Overlap between $W/Zq\bar{q}$+jets ($q=b,c$) 
events generated from the matrix element calculation and those
generated from parton-shower evolution in the $W/Z$+light jets
samples is avoided via an algorithm analogous to the one used
for $t\bar{t}$ Alpgen.


\subsubsection{Other backgrounds}\label{subsec:otherbkg}
%,tchanxs,Wtchanxs,schanxs}. 
Simulated samples of single top quark backgrounds corresponding to the
$s$-channel and $Wt$ production mechanisms are generated with {\sc
MC@NLO} v4.01~\cite{mcatnlo_1,mcatnlo_2,mcatnlo_3} using the {\sc
CT10} PDF set~\cite{ct10}.  In the case of $t$-channel single top
quark production, the {\sc AcerMC} v3.8 LO generator~\cite{acermc}
with the {\sc MRST LO**} PDF set is used.

Simulated samples of $t\bar{t}$ produced in association with a $W$ or $Z$ boson
($t\bar{t}V$ $(V=W,Z)$ in the following) are generated with the {\sc Madgraph v5} LO
generator~\cite{madgraph} and the {\sc CTEQ6L1} PDF set.  

Samples of $t\bar{t}$ produced in association with a Higgs boson
($t\bar{t}H$ in the following) are generated with the 
{\sc Pythia} 6.425~\cite{py6} LO generator and the {\sc MRST LO**} PDF set~\cite{mrst},
assuming a Higgs boson mass of $125\gev$ and considering the 
$H\to b\bar{b}$, $c\bar{c}$, $gg$, and $W^+W^-$ decay modes.

Parton shower and fragmentation are modelled with {\sc Herwig}
v6.520~\cite{herwig} in the case of {\sc MC@NLO}, with {\sc Pythia}
6.421 in the case of {\sc AcerMC}, and with {\sc Pythia} 6.425 in the
case of {\sc Madgraph}.  All these samples are generated assuming a top
quark mass of $172.5\gev$. The single top quark samples are normalised to
the approximate NNLO theoretical cross sections~\cite{stopxs,stopxs_2}
using the {\sc MSTW2008} NNLO PDF set, while the $t\bar{t}V$ samples
are normalised to the NLO cross section predictions~\cite{ttbarVxs1,ttbarVxs2}.
The $t\bar{t}H$ sample is normalised using the NLO theoretical cross section 
and branching ratio predictions~\cite{lhcxs}.
Finally, the diboson backgrounds are modelled using {\sc Herwig} with
the {\sc MRST LO**} PDF set, and are normalised to their NLO
theoretical cross sections~\cite{dibosonxs}.

\subsubsection{Signal samples}\label{subsec:MCsignal}


For vector-like $T$ signals, samples corresponding to a singlet $T$ quark 
decaying to $Wb$, $Zt$ and $Ht$ are generated with the {\sc Protos} v2.2 
LO generator~\cite{jaas,protos} 
using the  {\sc MSTW2008} LO PDF set, and interfaced to {\sc Pythia} for 
the parton shower and fragmentation. 

For each decay channel ($Wb$, $Zt$ and $Ht$) the branching ratio has been 
set to 1/3. Events are reweighted
in order to reproduce any desired branching ratio configuration. 

The predicted branching ratios in the weak-isospin singlet and doublet scenarios as 
a function of $m_{T}$ are given in Table~\ref{tab:BRT}.

The $m_{T}$ values considered range from $350\gev$ to $850\gev$ in steps of $50\gev$, 
with the Higgs boson mass assumed 
to be $125\gev$. All Higgs boson decay modes are considered, 
with branching ratios as predicted by {\sc hdecay}~\cite{hdecay}.

Signal samples are normalized to the approximate NNLO theoretical cross sections~\cite{ttbarxs} using the {\sc MSTW2008} NNLO PDF set.
The cross section values used are summarized in Table~\ref{tab:sigmaTT}.



\begin{table}[h!]
\begin{center}
\begin{tabular}{c c c c c c c}
\hline
\hline
 & \multicolumn{3}{c}{Singlet} &  \multicolumn{3}{c}{Doublet} \\
 $m_{T}$ ($\gev$) & $BR(T \to Wb)$ & $BR(T \to Zt)$ & $BR(T \to Ht)$ & $BR(T \to Wb)$ & $BR(T \to Zt)$ & $BR(T \to Ht)$\\
\hline
350 	&  0.545 	&  0.116 	&  0.338	&  0.000 	&  0.255 	&  0.745 	\\ 
400 	&  0.513 	&  0.139 	&  0.348	&  0.000 	&  0.285 	&  0.715 	\\
450 	&  0.502 	&  0.158 	&  0.341	&  0.000 	&  0.316 	&  0.684 	\\ 
500 	&  0.497 	&  0.173 	&  0.330	&  0.000 	&  0.343 	&  0.657 	\\
550 	&  0.495 	&  0.185 	&  0.321	&  0.000 	&  0.365 	&  0.635 	\\
600 	&  0.494 	&  0.194 	&  0.312	&  0.000 	&  0.383 	&  0.617 	\\ 	
650 	&  0.494 	&  0.202 	&  0.304	&  0.000 	&  0.399 	&  0.601 	\\ 
700 	&  0.494 	&  0.208 	&  0.298	&  0.000 	&  0.411 	&  0.589 	\\ 
750 	&  0.494 	&  0.214 	&  0.292	&  0.000 	&  0.422 	&  0.578 	\\ 
800 	&  0.494 	&  0.218 	&  0.288	&  0.000 	&  0.431 	&  0.569 	\\
850 	&  0.494 	&  0.222 	&  0.284	&  0.000 	&  0.439 	&  0.561 	\\ 
\hline
\hline
\end{tabular}
\caption{\label{tab:BRT} Branching ratios for $T$ decay as a function
of $m_{T}$ as computed with {\sc Protos} in the weak-isospin singlet and doublet scenarios.}
\end{center}
\end{table}
%%%%%%%%
\begin{table}[h!]
\begin{center}
\begin{tabular}{c c c c c}
\hline
\hline
 $m_{T}$ ($\gev$) & $\sigma(TT)$ (pb) & Scale uncertainties (pb) & PDF+$\alpha_s$ uncertainties (pb) & Total uncertainty (pb)\\
\hline
350 	&  5.083 		&  +0.140/-0.285 		&  + 0.569/-0.488 		&  +0.586/-0.565		\\
400 	&  2.296 		&  +0.066/-0.130 		&  + 0.269/-0.221 		&  +0.277/-0.257		\\
450 	&  1.113 		&  +0.034/-0.063 		&  + 0.136/-0.107 		&  +0.140/-0.125		\\
500 	&  0.5702 		&  +0.0185/-0.0327 		&  + 0.0723/-0.0545	 	&  +0.0746/-0.0636		\\
550 	&  0.30545 	&  +0.01040/-0.01769 	&  + 0.04012/-0.02889 	&  +0.0414/-0.0339		\\
600 	&  0.1696 		&  +0.0060/-0.0099 		&  + 0.0230/-0.0161	 	&  +0.0238/-0.0189		\\	
650 	&  0.09707 	&  +0.00359/-0.00571 	&  + 0.01363/-0.00936 	&  +0.01410/-0.01097	\\
700 	&  0.05694 	&  +0.00218/-0.00338 	&  + 0.00828/-0.00559 	&  +0.00856/-0.00653	\\
750 	&  0.03411 	&  +0.00135/-0.00204 	&  + 0.00513/-0.00343 	&  +0.00530/-0.00400	\\
800 	&  0.02080 	&  +0.00085/-0.00126 	&  + 0.00329/-0.00216 	&  +0.00340/-0.00250	\\
850 	&  0.01287 	&  +0.00054/-0.00079 	&  + 0.00215/-0.00138 	&  +0.00222/-0.00159 	\\
\hline
\hline
\end{tabular}
\caption{\label{tab:sigmaTT} Theoretical cross section at NNLO  for $TT$ production as a function
of $m_{T}$ as computed by {\sc Hathor}, and scale and PDF uncertainties.}
\end{center}
\end{table}
%%%%%%%%

\subsection{Multi-jet background}\label{sec:qcdbkg}

The contribution to the background from multijet events is
estimated via data-driven methods~\cite{ttbar_3pb}.


  For the $W$+jets
background, predictions for the shape of kinematic variables are obtained from the simulation but the
normalisation is determined from the data, using the predicted asymmetry between
$W^+$+jets and $W^-$+jets production in $pp$ collisions~\cite{wasym}, and 
separating the events into categories based on the multiplicity of $b$ and $c$ jets.
Details of the estimation of the multijet and
$W$+jets backgrounds are given in Section~\ref{sec:DataDrivenBackground}.
The rest of the backgrounds, as well as the signal, are estimated from 
the simulation and normalised using their theoretical cross sections. 


%\section{Object definition}\label{sec:objects}


\section{Search for \TTbar\ pairs decaying to $Wb+X$}\label{sec:wbx}

\subsection{Boosted $W$ reconstruction}\label{subsec:boostedW}

\subsection{Control regions}\label{sec:wbxCR}

\subsection{Event selection}\label{sec:wbxEVT}

%\subsection{}\label{sec:}

%\subsection{}\label{sec:}

\subsection{Systematics}\label{sec:wbxSYS}



\section{Preliminary search for \TTbar\ pairs decaying to $Ht+X$}\label{sec:htx}

\subsection{Control regions}\label{sec:htxCR}

\subsection{Event selection}\label{sec:htxEVT}

%\subsection{}\label{sec:}

%\subsection{}\label{sec:}

\subsection{Systematics}\label{sec:htxSYS}

