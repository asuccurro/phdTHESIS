All results include both statistical and systematic uncertainties,
and the consistency of the data with the background prediction is 
assessed by computing the $p$-value under the background-only hypothesis
(1-$CL_{\rm b}$) for each point of the two-dimensional plane 
(each point corresponding to a signal scenario) and for every heavy 
quark mass point considered (one two-dimensional plane is built for each
$m_T$ value).

We recall here that in the two-dimensional plane the gray area
corresponds to the unphysical region where the sum of BRs
exceeds unity. In every plane two benchmark models are indicated
as a plain circle and star symbols, corresponding respectively to the
weak-isospin singlet and doublet scenario BRs values as computed by the
\texttt{PROTOS} event generator.
To probe the full plane the signal samples are reweighted by the ratio
of desired branching ratio to the original branching ratio generated
by \texttt{PROTOS} and the complete analysis are repeated in each point.

